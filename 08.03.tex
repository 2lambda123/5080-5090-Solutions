{\bf 8.03}
3a.\\*
We know that\\
\indent$E(\bar{X})= \mu and E(S^{2})= \sigma ^{2} . $\\*
Therefore,\\
\indent$ E(2\bar{X}-5S^{2})= 2\mu - 5\sigma ^{2} .$\\
\\
Now we write things in terms of U and W:\\
\indent$ \bar{X}= \frac{U}{n}$\\


By properties of (8.2.8),\\
\indent$ S^{2}= \frac{W- \frac{1}{n} U^{2}}{n-1}= 
\frac{nW-U^{2}}{n(n-1)} .$\\

Therefore,\\
\indent$E(\underbrace{\frac{2U^{2}}{n}- 
\frac{5nW-5U^{2}}{n(n-1)}}_{unbiased estimator of 2\mu - 5\sigma ^{2} })= 
2\mu - 5\sigma ^{2}.$\\
\\*
3b.\\

We note that\\
\indent$E(X_{1}^{2})=\ldots E(X_{n}^{2})= \sigma ^{2}+ \mu ^{2}. 
$\\

Therefore,\\
\indent$ E(W)= n(\sigma ^{2}+ \mu ^{2})$;That is, 
$E(\frac{W}{n})\sigma ^{2}+ \mu ^{2} . $\\

So, \indent$\frac{W}{n}$ is an unbiased estimator of $\sigma ^{2}+ \mu ^{2}$.\\*
3c.\\

In terms of indicator functions, \indent$Yi = I\lbrace Xi \le c\rbrace . $\\

Therefore, \indent$E(Yi)=P\lbrace Xi \le c\rbrace = FX(c)= \Phi (\frac{c - 
?}{\sigma }),$\\

showing that $Y_{i}$ is an unbiased estimator for $F_{X}(c).$


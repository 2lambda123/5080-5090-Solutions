
\text{\huge Solutions 10.17}\\

\text{\Large 17(a)}\\

With a two parameter exponential EXP(1,$\eta$) the pdf is\\

$\quad f(x) = e^{-(x-\eta)}I[x > \eta]$ \\

$\quad = \enskip e^{-x}e^{\eta}I[x > \eta]$ \\
\[
$$This is a range-dependent exponential. Following the range-dependent theorem, stating if $q_1(\theta)$ is increasing and $q_2(\theta)$ is decreasing then $T_1 = min[q_1^{-1},q_2^{-1}]$ is a sufficient statistic. Therefore S=$X_{1:n}$ is complete and sufficient for $\eta$.$$ \\
\]

\text{\Large 17(b)}\\

In order to prove $X_{1:n}$ - $\frac{1}{n}$ is a UMVUE of $\eta$ we need to show it is unbiased.\\
\[
$$Note: If $X_i \sim$ iid f(x;$\theta$), S is complete and sufficient, and T=t(s) is unbiased, then T is a UMVUE.$$\\
\]
bias is given by b(T) = E(T) - $\tau{\theta}$\\

$E(X_{1:n}) = \eta + \frac{1}{n}$\\

$b(\eta) = \eta + \frac{1}{n} - \eta = \frac{1}{n}$\\
\[
$$Therefore $X_{1:n} - \frac{1}{n}$ is an unbiased estimator and thus a UMVUE for $\eta$ $$\\
\]

\text{\Large 17(c)} \\

To find the pth percentile ($x_p$) in terms of $\eta$ we need to solve\\

$ p= P(X_1 < x_p) = \int_{\eta}^{x_p} e^{-(x-\eta)} dx = \begin{cases} 1-e^{-(x_p - \eta)} \enskip if \enskip x_p > \eta \\
0, \enskip o/w \end{cases}$ \\

Therefore, $p=1-e^{-(x_p - \eta)} \enskip \Rightarrow \enskip e^{-(x_p - \eta)} = 1-p$\\

Hence, $log(e^{-(x_p - \eta)}) = log(1-p) \enskip \Rightarrow \enskip -x_p + \eta \enskip = \enskip log(1-p)$\\

Therefore, $x_p \enskip = \enskip \eta - log(1-p)$\\

For the UMVUE of $x_p$, find a function of S that is unbiased.\\

It follows that $E(\hat{x_p}) = \eta - log(1-p)$\\

Using Part(b) we get $\hat{x_p} \enskip = \enskip X_{1:n} - \frac{1}{n} - log(1-p)$\\